\documentclass[a4paper, 12p]{article}

\begin{document}

\author{Carlo Allietti}
\title{Explaining Reverse Chain Rule}
\date{\today}


\maketitle

\section{The Equation}

Here is a more consise form of the equation for reverse chain rule.

\[\int f \circ g dx = \int \frac{f \circ g}{g'} dg\]

\section{Using My Method}

Instead of defining a $u$, we do the same process of finding the ``inner'' function, but set that as the expression you are integrating by, rather than first defining it as $u$.

\[\int x \sin x^{2} dx \rightarrow \int x \sin x^{2} d x^{2}\]

This is essentially the same as.

\[\int x \sin x^{2} dx \rightarrow \int x \sin u du\]

Then you divide the expression being integrated by the derivative of the inside function.

\[\int x \sin x^{2} dx^{2} \rightarrow \int \frac{x \sin x^{2}}{2x} dx^{2} \]

Which is essentially the same as.

\[\int x \sin u du \rightarrow \int \frac{x \sin u}{2x} du \]

Next you simplify.

\[\int \frac{x \sin x^{2}}{2x} dx^{2} \rightarrow \frac{1}{2}\int \sin x^{2} dx^{2} \]

And finish integrating, keeping in mind that you are integrating by the ``inner'' function, not $x$.

\[\frac{1}{2}\int \sin x^{2} dx^{2} \rightarrow \frac{1}{2}(- \cos x^{2}) \]

In summary, notice how little we had to actually write to integrate.
\[\int x \sin x^{2} dx\]
\[\int \frac{x \sin x^{2}}{2x} dx^{2}\]
\[\int \frac{1}{2} \sin x^{2} dx^{2}\]
\[-\frac{1}{2} \cos x^{2}\]

\section{Understanding My Method}
Perhaps you might agree that this method saves time on some problems. But you might wonder if this method comes at the cost of understanding, making someone more confused. I would argue that this method has the potential to do the opposite. Let me show how this function is extracted from the chain rule.
\[(f(g(x)))' = f'(g(x))g'(x)\]
Now we are going to take the antiderivative of $f$, the rule is still the same, just easier for us to work with since now the we have an $f$ on the right side rather than an $f'$.
\[(\int f(g(x)) dg(x))' = f(g(x))g'(x)\]
Notice that we are integrating by $g(x)$, not by $x$, this is since we only took the antiderivative of $f$, NOT $g$. We are basically composing the functions $\int f(u) du$ (which is the antiderivative of $f$) and $g(x)$ to get $\int f(g(x)) dg(x)$.

Now we are going to replace $f$ with $f/g'(x)$, so that after we simplify, we can get $f(g(x))$ on one side of the equation.
\[(\int \frac{f(g(x))}{g'(x)} dg(x))' = \frac{f(g(x))}{g'(x)}g'(x)\]
\[(\int \frac{f(g(x))}{g'(x)} dg(x))' = f(g(x))\]

Now take the antiderivative on both sides. This cancels out with the derivative on the left side, which leaves us with:
\[\int \frac{f(g(x))}{g'(x)} dg(x) = \int f(g(x)) dx\]

That is the reverse chain rule.

Now lets prove that this is truely the reverse of the chain rule by taking the derivative of both sides.
\[(\int \frac{f(g(x))}{g'(x)} dg(x))' = f(g(x))\]

If this is really the reverse of the chain rule you would expect applying the chain rule to the left side will result in $f(g(x))$.
Of course the ``inner'' function on the left side is $g(x)$ and the ``outer'' function is $\int \frac{f(u)}{g'(x)} du$. (The derivative of the outer function should be $\frac{f(u)}{g'(x)}$)

\[\frac{f(g(x))}{g'(x)}g'(x) = f(g(x))\]

Which simplifies.
\[f(g(x)) = f(g(x))\]

I am not adding $+C$ because screw you.
\end{document}
